\lawchapter{Asamblea y Campamento Anual}

\lawsection{Sección Única}

\lawsubsection{De los delegados}

\article
\label{participacion-eventos}
Los delegados a la Asamblea Anual deben tener como mínimo un año como miembros de la organización. Los demás participantes del Campamento deben tener, como mínimo, seis meses como miembros de la organización. Para ambos eventos es indispensable contar con el aval del Pastor.

\article
\label{delegados-asamblea}
Se elegirá como delegado a la Asamblea un joven por cada Liga Local. Todos los miembros del Consejo Conferencial son delegados a la Asamblea por derecho propio y no utilizan la capacidad de la Liga Local a la que pertenecen.

\lawsubsection{De la apertura}

\article
\label{eleccion-presidente-sustituto}
Llegada la hora convenida para la reunión, el Presidente llamará al orden. Si a la hora convenida el Presidente no estuviese presente, ocupará la presidencia el Vicepresidente. Si este no se encontrara presente, el secretario dirigirá a la Asamblea Anual en la elección de un Presidente provisional.

\article
Toda sesión será comenzada con un devocional. Seguidamente el Presidente pedirá el pase de lista y la comprobación de que esté presente la mitad más uno de los delegados a la Asamblea, que constituye el quórum requerido para la misma. El Presidente dará inicio a la sesión en el nombre de nuestro Señor Jesucristo, proclamando estas palabras: «En el nombre de nuestro Señor Jesucristo declaro abierta esta sesión».

\article
Abierta la sesión, el Presidente pedirá se dé lectura al acta de la Asamblea anterior, con el objetivo de revisar los acuerdos tomados y su cumplimiento, lo que se someterá a consideración de la Asamblea. Seguidamente, se dará lectura al orden del día que deberá ser aprobado por la misma.

\article
El orden del día de la Asamblea deberá incluir entre sus puntos la elección de dos comisiones revisoras de libros: una para auditar los libros de Tesorería, y otra para auditar los libros de Secretaría.

\lawsubsection{De la conducción de la Asamblea}

\article
Durante la Asamblea, el Presidente:
\begin{enumerate}
    \item Concederá el uso de la palabra al que la pidiera debidamente.
    \item Velará que la discusión se conduzca con el debido respeto los unos por los otros.
    \item No concederá la palabra a nadie, una vez que se haya hecho una proposición, hasta que esta haya sido secundada, y en caso de no serlo, hasta que se haya hecho la aclaración al efecto.
    \item Retirará el uso de la palabra al asambleísta que faltare el respeto a otro en la Asamblea, y si lo creyere conveniente, lo someterá a votación para que los asambleístas decidan si debe seguir o no en la Asamblea.
    \item No interrumpirá al que esté haciendo uso de la palabra, a menos que quebrante las reglas.
    \item No mostrará parcialidad con gestos, realizando o diciendo algo que pueda hacer que la Asamblea favorezca o rechace lo que está discutiendo.
    \item No deberá excitarse, ni ser descortés o autoritario.
    \item No dirá «yo creo», «yo pienso», sino, «esta presidencia cree», «esta presidencia piensa».
    \item Pedirá al Vicepresidente, o cuando este no estuviese presente, a un Presidente provisional elegido según el artículo \ref{eleccion-presidente-sustituto}, que ocupe su puesto, a fin de tomar parte de la discusión.
    \item Decidirá sobre cuestiones de orden, a menos que él desee que la Asamblea decida, en cuyo caso tomará esa iniciativa.
\end{enumerate}

\article
Durante la Asamblea, los delegados:
\begin{enumerate}
    \item Solicitarán la palabra en su oportunidad. Si varios asambleístas solicitaran la palabra en la misma oportunidad, se le concederá al que esté más distante de la presidencia.
    \item No podrán hablar sin pedir y obtener el uso de la palabra.
    \item Apelarán a la Asamblea en cualquier momento en que se considere vejado por el Presidente o perjudicado por una decisión de este.
    \item No podrán interrumpir al que esté hablando, ni hacer alguna clase de manifestación, a menos que sea una cuestión de orden, o para rectificar una alusión personal o a un ausente.
    \item No podrán sostener diálogo ni se podrán dirigir a otro sin la autorización de la presidencia.
    \item Deberán obedecer las indicaciones de la presidencia.
    \item Deberán aceptar la decisión de la mayoría una vez realizada la votación, y en ningún caso podrán censurarla.
    \item No deberán excitarse, ni faltarse el respeto mutuamente, sino conducirse de manera digna y decorosa.
\end{enumerate}

\lawsubsection{De las mociones y proposiciones}

\article
Solo podrán hacer mociones y proposiciones los delegados a la Asamblea, y para que estas puedan discutirse, es necesario que sean secundadas.

\article
Una vez secundada una moción o proposición, el Presidente dirá: «Ha sido propuesto y secundado que\ldots se pone a discusión el asunto.»

\article
Para retirarse una moción o proposición, tendrá que hacerse con el consentimiento de quién la secundó, pero si ya hubiese comenzado la discusión, tendrá que hacerse con el consentimiento de la Asamblea.

\article
Se concederán en la discusión dos turnos para hablar a favor y dos turnos para hablar en contra, de manera alternada.

\article
Podrá hacerse una enmienda o proposición relacionada con el asunto original. También podrá realizarse una segunda enmienda relacionada también con la proposición original, la cual estará sujeta a las mismas reglas. Una moción para enmendar una segunda enmienda estará fuera de orden. La primera enmienda se pone a votación antes que la proposición original y la segunda enmienda antes que la primera. Si procede la enmienda, la moción original queda enmendada y en esa forma se pondrá a votación.

\article
Cuando se está discutiendo una moción o proposición podrán admitirse algunas de las siguientes proposiciones, pero no podrán hacerse ni ponerse a votación enseguida:
\begin{enumerate}
    \item Levantar la sesión si se ha pasado la hora reglamentaria.
    \item Prorrogar la sesión.
    \item Cuestión de orden.
    \item Cuestión previa.
    \item Ampliación de la discusión.
    \item Dejar asunto sobre la mesa.
    \item Referir el asunto a una comisión.
    \item Limpiar la mesa.
    \item Dividir la moción.
    \item Cerrar el debate.
\end{enumerate}
Si se hacen más de una de estas proposiciones, se votará según el orden expresado más arriba. Ninguna persona que estuviese hablando podrá hacer las mociones de: Levantar la sesión, cuestión previa y dejar el asunto sobre la mesa.

\article
Se propondrá prorrogar la sesión cuando, llegada la hora reglamentaria para terminar, se estuviere tratando o hubiere algún asunto de interés que considerar.

\article
La cuestión de orden es una proposición que tiene por objeto llamar la atención del Presidente acerca de alguna infracción de esta regla, o traer al orador al asunto que se está tratando.

\article
La proposición de la cuestión previa tiene por efecto cerrar la discusión y traer a la Asamblea la decisión inmediata, por estimarse que el asunto ha sido ampliamente discutido.

\article
Cuando se desee una discusión más amplia, se propondrá que se amplíe la misma en un número determinado de turnos.

\article
Cuando se aplaza el debate para más adelante o para otra sesión, se podrá proponer que el asunto se deje sobre la mesa hasta entonces.

\article
Cuando se desee evitar pérdida de tiempo y energía a la Asamblea, se podrá proponer que la proposición pase al Consejo Conferencial para su estudio e informe a la Asamblea en su próxima sesión.

\article
Cuando se desee que un asunto pase al estudio de personas que tengan cierta facilidad o capacidad para resolverlo, se propondrá que pase a una comisión especial que, formada por esas personas, informará a la Asamblea en la próxima sesión.

\article
Cuando haya una discusión y no sea fácil volver al asunto debidamente, cualquier asambleísta podrá proponer que se limpie la mesa. Si se aprueba esa proposición, podrán hacerse enseguida nuevas proposiciones sobre el asunto.

\article
Si una proposición se está discutiendo en varias partes, los asambleístas podrán proponer que sea dividida, y que se tome decisión separada sobre cada parte. Si es aprobada la separación, podrán presentarse las enmiendas a todas o cualquiera de las partes, igual que si se tratara de la proposición original.

\article
La proposición de cerrar el debate tiene por efecto concluir la discusión actual cuando no existan más proposiciones.

\lawsubsection{De la discusión y el debate}

\article
El asambleísta que durante la discusión se considere aludido, con permiso del Presidente, podrá pedir una rectificación, limitándose solo a la alusión. Si la alusión se dirige hacia una persona ausente, cualquier asambleísta pedirá la palabra para defenderlo. Esta rectificación no se considera como un turno reglamentario en la discusión.

\article
Cualquier asambleísta tiene el derecho durante la discusión a pedir la lectura de documentos que se relacionen con el asunto, sin que ello signifique que utilice un turno reglamentario.

\article
Cuando se vaya a discutir un asunto sobre el cual se supone que todos estarán de acuerdo y se quiere ahorrar tiempo de la Asamblea, cualquier asambleísta podrá proponer la supresión temporal de los preceptos reglamentarios de la discusión, que será aprobado si votan a favor las dos terceras partes de los asambleístas.

\article
Podrá hacerse la revisión de un acuerdo tomado en la misma sesión. La revisión será aprobada si obtiene el voto de las dos terceras partes de la Asamblea.

\article
Todas los acuerdos tomados por la Asamblea, después de ser aprobados, serán modificados únicamente si las proposiciones se realizan en la misma sesión en que estos fueron aprobados.

\lawsubsection{De las Votaciones}

\article
\label{votaciones-inicio}
El método de votación será elegido ordinariamente por el Presidente, pero la Asamblea podrá acordar el método que crea conveniente en un caso determinado.

\article
Métodos de Votación:
\begin{enumerate}
    \item Levantando la mano cuando se trata de asuntos corrientes.
    \item Ponerse de pie cuando se quiere tener más seguridad en el cómputo de los votos.
    \item Por papeletas o boletas cuando se trata de asuntos personales, asuntos que requieren una votación secreta, en caso de elecciones, o cuando surgen dificultades en el cómputo de los votos.
    \item Votación nominal cuando se desea obtener el parecer personal de cada uno de los asambleístas en alta voz.
\end{enumerate}
Cuando el método de votación elegido requiera que las propuestas sean votadas en orden, se someterán a votación en el orden inverso al que fueron presentadas.

\article
El quórum requerido para la votación será la mitad más uno de los delegados a la Asamblea, sin el cual no se podrá realizar la votación.

\article
Tienen derecho a emitir el voto solo los delegados que hayan permanecido todo el tiempo en la sesión. Al terminar la votación el Presidente dirá: «Queda cerrada la votación».

\article
El Presidente solo podrá votar cuando se use el método de votación por papeletas o boletas, o en caso de empate para decidir.

\article
Para que no haya dificultades en la votación deberán sentarse separados los delegados y los visitantes.

\label{votaciones-final}
