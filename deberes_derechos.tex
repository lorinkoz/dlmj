\lawchapter{Deberes y derechos de los miembros}

\lawsection{Sección Única}

\article
Los miembros de la \LMJ{} tendrán los siguientes derechos:
\label{derechos-miembros}
\begin{enumerate}[a)]
    \item Participar en todas las actividades de la liga, ya sean locales, distritales o nacionales, teniendo en cuenta lo relacionado con el artículo \ref{participacion-eventos}.
    \item Elegir y ser elegido para todos los cargos de la liga, siempre y cuando cumpla los requisitos establecidos para el cargo.
    \item Representar, elegir y ser electo delegado a cualquier actividad organizada por el Consejo Conferencial, siempre y cuando cuente con el aval de su Pastor.
    \item Hacer uso de la palabra y del voto en las Reuniones de Planeamiento.
    \item Conocer la Disciplina de la \LMJ{} y participar en sus modificaciones.
\end{enumerate}

\article
\label{deberes-miembros}
Los miembros de la \LMJ{} tendrán los siguientes deberes:
\begin{enumerate}[a)]
    \item Seguir a Cristo y dar testimonio de su fe.
    \item Asistir regularmente a las reuniones y activiades de la liga, así como a los cultos de la iglesia.
    \item Cumplir con las tareas que le asigne la iglesia y la liga.
    \item Cumplir en tiempo y forma con la asignación conferencial.
\end{enumerate}

\article
Cuando un joven no cumple con sus deberes como miembro de la \LMJ{} se trabajará en su recuperación espiritual. Si mantiene la misma actitud, deberá ser requerido en el amor de Cristo por el Consejo Local. De no existir cambios en su conducta, se le dará baja de la membresía de la \LMJ{}, con el voto de las dos terceras partes del Consejo Local y la observancia del Pastor de la iglesia.
