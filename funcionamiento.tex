\lawchapter{Composición, funcionamiento y trabajo}

\lawsection{Sección Primera: Nivel Local}

\article
Para constituir una \LMJ{} local deben existir tres miembros como mínimo, de los cuales uno debe ser miembro de la iglesia.

\article
\label{membresia-liga}
Para llegar a ser miembro de la \LMJ{} se requiere asistir regularmente a la iglesia por seis meses como mínimo, y gozar de buen testimonio.

\article
\label{amigo-liga}
Para llegar a ser amigo de la \LMJ{} se requiere asistir a la iglesia regularmente por tres meses como mínimo, y gozar de buen testimonio.

\article
\label{identidad-sexual}
Los miembros y amigos de la \LMJ{} asumen la identidad sexual según el diseño original de Dios, expresado en Génesis 1:27.

\article
La \LMJ{} estará dirigida por un Consejo Local, formado por: Presidente, Vicepresidente, Secretario, Tesorero, Estadístico, Director de Publicidad, Director de Programas y un Asesor.
Aquellas ligas que por falta de miembros no puedan tener un Consejo Local, podrán trabajar con la elección de los cargos que les sean más convenientes.

\article
Los miembros del Consejo Local tendrán las siguientes funciones:

\articlepart{Presidente}
\begin{enumerate}[noitemsep]
    \item Presidirá todas las reuniones del Consejo Local, referidas en el artículo \ref{reuniones-locales}.
    \item Promoverá e impulsará el trabajo de la liga.
    \item Representará a la liga en la Junta Local de Trabajo.
    \item Representará a la liga en el ámbito local, distrital y nacional, ante todos los órganos de la misma.
\end{enumerate}

\articlepart{Vicepresidente}
\begin{enumerate}[noitemsep]
    \item Presidirá las reuniones en ausencia del Presidente.
    \item Sustituirá al Presidente en la Junta Local de Trabajo en ausencia de este.
    \item Ocupará el cargo de Presidente si este tuviese que recesar sus funciones.
\end{enumerate}

\articlepart{Secretario}
\begin{enumerate}[noitemsep]
    \item Redactará las actas de las reuniones del Consejo Local.
    \item Mantendrá actualizado el registro de miembros y amigos de la liga.
    \item Atenderá toda la correspondencia de la Liga.
    \item Citará a los miembros del Consejo Local para las reuniones.
\end{enumerate}

\articlepart{Tesorero}
\begin{enumerate}[noitemsep]
    \item Recaudará y guardará los fondos de la liga.
    \item Anotará cuidadosamente las entradas y salidas de dinero en el libro habilitado.
    \item Confeccionará un balance mensual de los ingresos y egresos de la liga y presentará el informe del estado de las cuentas en las reuniones planificadas.
    \item Preparará las entregas de asignación a la Tesorería Nacional.
    \item Mantendrá orientada a la liga acerca de su asignación.
\end{enumerate}

\articlepart{Estadístico}
\begin{enumerate}[noitemsep]
    \item Llevará un registro de asistencia de los miembros de la liga a cultos y demás actividades planificadas.
    \item Realizará el control estadístico de los miembros de la liga, actualizando la información periódicamente.
    \item Enviará en tiempo y forma la información al Estadístico Nacional o al Promotor de Distrito, según le sea solicitada.
\end{enumerate}

\articlepart{Director de Publicidad}
\begin{enumerate}[noitemsep]
    \item Dará publicidad a las reuniones de la liga, actividades locales y distritales, así como proyectos y eventos conferenciales.
    \item Servirá de corresponsal de la Liga Local ante los medios de prensa de la iglesia y el \OOLMJ{}.
\end{enumerate}

\articlepart{Director de Programa}
\begin{enumerate}[noitemsep]
    \item Realizará el programa de cultos y actividades de la Liga Local.
    \item Promoverá el trabajo de los jóvenes interrelacionando las diferentes áreas y ministerios de la iglesia.
    \item Nombrará comisiones o áreas de trabajo bajo su responsabilidad, cuyos miembros no tendrán derecho a participar en las reuniones del Consejo Local.
\end{enumerate}

\articlepart{Asesor}
\begin{enumerate}[noitemsep]
    \item Aconsejará, exhortará, animará, reprenderá y orientará a los jóvenes, conduciéndolos por el camino del Señor.
    \item Asistirá a las reuniones regulares de la liga y el Consejo Local.
    \item Cooperará eficazmente en la ejecución de los planes de la Liga Local.
\end{enumerate}

\article
\label{reuniones-locales}
Las reuniones de la Liga Local serán las siguientes:

\articlepart{Reuniones del Consejo Local}
\begin{enumerate}[noitemsep]
    \item Participarán todos los miembros del Consejo Local.
    \item Se realizarán una vez al mes como mínimo, en el local y la hora que el Consejo determine.
\end{enumerate}

\articlepart{Reuniones de Planeamiento}
\begin{enumerate}[noitemsep]
    \item Participarán todos los miembros de la Liga Local.
    \item Se realizarán cada dos meses como mínimo.
    \item Los miembros del Consejo informarán sobre todos los trabajos y planes a realizar.
    \item Se realizarán las elecciones de los miembros del Consejo Local, así como de los delegados a la Asamblea Anual y demás actividades, según corresponda.
\end{enumerate}

\articlepart{Reuniones extraordinarias}
Serán convocadas por el Presidente o el Asesor, para tratar cualquier asunto de carácter urgente.

\lawsection{Sección Segunda: Nivel Nacional}

\article
Habrá un Consejo Conferencial que planeará el Programa General de la \LMJ{}, orientando a la juventud en pos de alcanzar la estatura del varón perfecto. El Consejo Conferencial estará formado por: Presidente, Vicepresidente, Secretario, Tesorero, Estadístico, Director de Publicidad, Director de Programa, Director del \OOLMJ{}, Promotores de Distrito y Asesor Conferencial.

\article
\label{reunion-consejo-conferencial}
El Consejo Conferencial se reunirá una vez al año como mínimo, en el lugar y fecha acordados.

\article
El Consejo Conferencial convocará y organizará la Asamblea y Campamento Anual de la \LMJ{}.

\article
Los delegados que representarán a la \LMJ{} en eventos internacionales serán elegidos por el Consejo Conferencial.

\article
Los miembros del Consejo Conferencial tendrán las siguientes funciones:

\articlepart{Presidente}
\begin{enumerate}[noitemsep]
    \item Dirigirá el trabajo de todos los miembros del Consejo Conferencial, apoyado en el Asesor.
    \item Presidirá las reuniones del Consejo y la Asamblea Anual.
    \item Representará a la \LMJ{} en la Junta Consultiva.
    \item Representará a la liga ante todos los organismos clericales y seculares, nacionales e internacionales.
\end{enumerate}

\articlepart{Vicepresidente}
\begin{enumerate}[noitemsep]
    \item Presidirá las reuniones del Consejo Conferencial y la Asamblea Anual en ausencia del Presidente, o cuando este lo solicite.
    \item Sustituirá al Presidente en la Junta Consultiva en ausencia de este.
    \item Ocupará el cargo de Presidente si este tuviese que recesar sus funciones.
\end{enumerate}

\articlepart{Secretario}
\begin{enumerate}[noitemsep]
    \item Citará a los miembros del Consejo Conferencial para las reuniones, en consulta con el Presidente y el Asesor.
    \item Redactará las actas de las reuniones del Consejo Conferencial.
    \item Enviará un resumen escrito de los acuerdos de cada reunión a todos los miembros del Consejo Conferencial.
\end{enumerate}

\articlepart{Tesorero}
\begin{enumerate}[noitemsep]
    \item Recaudará y guardará los fondos nacionales de la liga.
    \item Anotará cuidadosamente las entradas y salidas de dinero en el libro habilitado.
    \item Confeccionará los comprobantes de entrada y salida con las autorizaciones pertinentes.
    \item Rendirá informe del estado de las cuentas en las reuniones planificadas.
    \item Mantendrá orientadas a la ligas y Promotores de Distrito acerca de su asignación.
\end{enumerate}

\articlepart{Estadístico}
\begin{enumerate}[noitemsep]
    \item Recopilará y procesará toda la información estadística de las Ligas Locales.
    \item Mantendrá orientadas a la ligas y Promotores de Distrito acerca de su información estadística.
\end{enumerate}

\articlepart{Director de Publicidad}
\begin{enumerate}[noitemsep]
    \item Dará publicidad a todos los programas, actividades y eventos conferenciales de la \LMJ{}.
    \item Nombrará comisiones o áreas de trabajo bajo su responsabilidad, cuyos miembros no tendrán derecho a participar en las reuniones del Consejo Conferencial.
    \item Será miembro del Consejo de Redacción del \OOLMJ{}.
\end{enumerate}

\articlepart{Director de Programa}
\begin{enumerate}[noitemsep]
    \item Supervisará el desarrollo del programa de todos los cultos y actividades conferenciales de la \LMJ{}.
    \item Nombrará comisiones o áreas de trabajo bajo su responsabilidad, cuyos miembros no tendrán derecho a participar en las reuniones del Consejo Conferencial.
    \item Será miembro del Consejo de Redacción del \OOLMJ{}.
\end{enumerate}

\articlepart{Director del \OOLMJ{}}
\begin{enumerate}[noitemsep]
    \item Dirigirá y representará el \OOLMJ{} (Revista Compromiso) ante el Consejo Conferencial y la Asamblea Anual de la \LMJ{}.
    \item Seleccionará los miembros del Consejo de Redacción y los dará a conocer al Consejo Conferencial.
    \item Revisará, de conjunto con los miembros del Consejo de Redacción, los materiales a publicar para su aprobación.
\end{enumerate}

\articlepart{Promotor de Distrito}
\begin{enumerate}[noitemsep]
    \item Desarrollará los planes del Consejo Conferencial en su distrito.
    \item Visitará todas las ligas de su distrito.
    \item Representará a la \LMJ{} en la Junta Asesora del distrito.
    \item Celebrará reuniones de orientación con los líderes de las Ligas Locales.
    \item Recaudará la asignación conferencial de su distrito y la remitirá al Tesorero Nacional.
    \item Llevará el registro de las ligas de su distrito y remitirá estos datos al Estadístico Nacional.
    \item Celebrará dos reuniones distritales al año como mínimo.
    \item Nombrará comisiones o áreas de trabajo bajo su responsabilidad, cuyos miembros no tendrán derecho a participar en las reuniones o eventos del Consejo Conferencial.
\end{enumerate}

\articlepart{Asesor Conferencial}
\begin{enumerate}[noitemsep]
    \item Aconsejará, exhortará, animará, reprenderá y orientará a los jóvenes, conduciéndolos por el camino del Señor.
    \item Asesorará, orientará y supervisará el trabajo del Consejo Conferencial.
    \item Estará presente en las reuniones del Consejo Conferencial.
    \item Cooperará eficazmente en la ejecución de los planes del Consejo Conferencial.
\end{enumerate}

\article
\label{informes-consejo}
Todos los miembros del Consejo Conferencial, a excepción del Asesor, presentarán informe ante la Asamblea Anual, donde resumirán todo el trabajo realizado durante el año conferencial.
